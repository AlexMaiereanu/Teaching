\begin{center}
\textbf{Important stuff that you should read carefully!}
\end{center}

\paragraph{State of these notes}
I constantly work on my lecture notes.
Therefore, keep in mind that:
\begin{compactitem}
\item I am developing these notes in parallel with the lecture---they can grow or change throughout the semester.
\item These notes are neither a subset nor a superset of the material discussed in the lecture.
\item Unless mentioned otherwise, all material in these notes is exam-relevant (in addition to all material discussed in the lectures).
\end{compactitem}
\medskip

\paragraph{Collaboration on these notes}
I am writing these notes using LaTeX and storing them in a git repository on GitHub at \url{https://github.com/florian-rabe/Teaching}.
Familiarity with LaTeX as well as Git and GitHub is not part of this lecture. But it is essential skill for you.
Ask in the lecture if you have difficulty figuring it out on your own.
\medskip

As an experiment in teaching, I am inviting all of you to collaborate on these lecture notes with me.
\medskip

By forking and by submitting pull requests for this repository, you can suggest changes to these notes.
For example, you are encouraged to:
\begin{compactitem}
\item Fix typos and other errors.
\item Add examples and diagrams that I develop on the board during lectures.
\item Add solutions for the homeworks if I did not provide any (of course, I will only integrate solutions after the deadline).
\item Add additional examples, exercises, or explanations that you came up or found in other sources.
 If you use material from other sources (e.g., by copying an diagram from some website), make sure that you have the license to use it and that you acknowledge sources appropriately!
\end{compactitem}
The TAs and I will review and approve or reject the changes.
If you make substantial contributions, I will list you as a contributor (i.e., something you can put in your CV).
\medskip

Any improvement you make will not only help your fellow students, it will also increase your own understanding of the material.
Therefore, I can give you up to $10\%$ bonus credit for such contributions.
(Make sure your git commits carry a user name that I can connect to you.)
Because this is an experiment, I will have to figure out the details along the way.

\paragraph{Other Advice}
I maintain a list of useful advice for students at \url{https://svn.kwarc.info/repos/frabe/Teaching/general/advice_for_students.pdf}.
It is mostly targeted at older students who work in individual projects with me (e.g., students who work on their BSc thesis).
But much of it is useful for you already now or will become useful soon.
So have a look.