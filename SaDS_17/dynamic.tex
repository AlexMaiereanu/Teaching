Dynamic analysis is a collective term for methods that evaluate code by executing it.

There is a variety of different dynamic methods, which should be discussed in a software engineering course.
A good overview is given in the corresponding chapters in the Software Engineering Body of Knowledge (SWEBOK) \cite{swebok}\footnote{Or the respective wikipedia pages, which are often summaries of the SWEBOK or similar sources.}.

The methods can be roughly classified into detection and diagnosis of faults.

\section{Detect Faults}

\textbf{Testing} is the most important dynamic analysis.
Testing can be classified using a number of different properties.
For achieving dependability, the following combination is most important:
\begin{compactitem}
 \item Unit testing: The tested component is small software unit like a class or a function.
 \item Conformance testing: The tested component is tested against its specification.
 This usually involves a set of test cases that are executed:
  \begin{compactitem}
   \item A test case for an implementation $f$ of a function specification $F$ is a pair $(x,F(x))$.
   The test case runs $f(x)$ and compares the result to $F(x)$.
   \item When testing a class $C$, a simple test case consists of constructor arguments $x$ and a sequence of test cases for methods of $C$.
   The test case runs $c:=\anew{C}{x}$ and then runs each test case on $c$.
   Additionally, the test may evaluate a class invariant after each step.
  \end{compactitem}
 \item Automated testing: The test cases are programmed in parallel to the implementation.
 Thus, they can be build and run automatically.
 \item Whitebox or blackbox testing: In the simplest case, testing is black-box, i.e., the test does not need to access the source code.
 However, whitebox testing can be valuable, especially if combined with a coverage check.
 For example, the test cases can be designed to execute every line of the source code, which can be checked with a coverage analysis tool.
\end{compactitem}
\medskip

\textbf{Coverage} checking traces program execution to determine which lines of the source code were entered.
In combination with an appropriately large set of test cases, this has a good chance of detecting control flow errors. 
\medskip

\textbf{Memory debugging} monitors memory (de)allocation and other dynamic memory operations.
This can help detect buffer overflows or memory leaks.

\section{Diagnose Faults}

If errors are confusing, it may be necessary to trace the program behavior step-by-step to locate the fault.

\textbf{Logging} is the simplest diagnostic technique. It prints of variables in specific places to detect exactly which commands computes a faulty value.

\textbf{Debugging} is a free form of logging. The program is executed step by step interactively, and the programmer can inspect variable values after every step.

\textbf{Profiling} watches the execution of all steps and accumulates which program unit uses how much time and/or memory.
 This can be used to find efficiency problems caused by design or implementation faults.


% https://en.wikipedia.org/wiki/Memory_debugger Valgrind, Purify
