Static analysis is a collective term for methods that do not execute the program.
It is performed by external tools that analyze the source code to detect faults.

Usually static analysis works with the source code.
But in principle it can work on the binary as well, e.g., some tools work on Java byte code.

Due to undecidability, detecting faults requires insight and careful case-by-case analysis, which is expensive.
Therefore, it has become very successful to focus on classes of faults and then to systematically find their occurrences.
We can think of these as individual heuristics, each hunting for a specific common bug, e.g., the ones discussed in Sect.~\ref{sec:sd:commonbugs}).
State-of-the-art tools check for hundreds of such fault classes.
Depending on the fault class, the checks may exhibit false-negatives (not all instances are detected) or false-positives (correct code is falsely marked).

\section{Work Flow}

\subsection{Compilation}

The first line of defense against faults is compilation, which is the simplest form of static analysis.

The primary goal of compilation is to translate the source code into another language, usually a binary file in a machine-near language such as assembly or byte code.
However, like all good algorithm operating on user input, compilers usually consist of three steps:
\begin{compactenum}
 \item parse the source files into a value $p$ of an abstract data type $L$ representing the context-free grammar of the source language
 \item check $p$ in order to verify the context-sensitive constraints of the source language
 \item process $p$ to obtain the compiled program, usually consisting of
   \begin{compactitem}
    \item optimization that transforms $p:L$ to $p':L$
    \item printing $p'$ in the target language
   \end{compactitem}
\end{compactenum}

Here the middle step is a static analysis.
It usually consists of at least
\begin{compactitem}
 \item scope-checking: ensure that only global or local identifiers are used that
  \begin{compactitem}
   \item have been declared
   \item are in scope
   \item are accessible (e.g., not private to a class)
  \end{compactitem}
 \item type-checking: ensure that only expressions are formed that are well-typed, i.e., check that
 \begin{compactitem}
  \item arguments of class constructors
  \item arguments of function and method invocations
  \item assignments to variables
  \item parameters of built-in language constructs (e.g., the condition of an if-statement)
  \end{compactitem}
  have the expected type
\end{compactitem}

Compilers in particular focus on checks that are necessary to ensure that printing out the compiled program succeeds at all.
For example, printing an ill-scoped program might cause a run-time error of the compiler. 

But compilers may carry out arbitrarily many additional static analysis checks.

\subsection{Additional Analysis}

Typically scope and type-checking (if successful) are followed by additional static analysis.
This can be done by
\begin{compactitem}
 \item the compiler itself
 \item compiler plugins
 \item separate tools that use the compiler for parsing and checking and then perform other analysis
 \item separate tools that use their own parser
\end{compactitem}
Some of these tools can be integrated with the IDEs to report the results in the same way as compiler errors.

\section{Individual Checks}

There is a wide variety of possible checks that can be implemented by static analysis tools.
An overview of tools for various programming languages can be found at \url{https://en.wikipedia.org/wiki/Static_code_analysis}
For example, findbugs is a popular tool working on Java byte code; it checks for over $400$ different faults.

In the following, we list some of the most important checks.

\paragraph{Coding Style}
The simplest static analysis is to check for coding style.
This is trivial but technically a special case of static analysis.

It often requires a separate parser because the compiler's parser may ignore the details of whitespace.

\paragraph{Common Error Patterns}
Many errors are made so often that it is worth performing a static analysis for them.

For example, in some programming languages $m\modop 2$ (arguably falsely) evaluates to $-1$ instead of $1$ for odd negative numbers.
Therefore, a static analysis tool may flag $m\modop 2==1$ as a likely error (because the expressions behaves unexpectedly for negative $m$).

\paragraph{Non-Exhaustive Match}
Languages with inductive data types allow performing exhaustiveness checks.
This checks whether there is a case for any possible value of the matched expression.
However, because the programmer might know from invariants that certain values are impossible, compilers languages often allow not giving cases for all values.

An exhaustiveness check flags these as likely errors.

\paragraph{Uninitialized Variables and Memory}
Some languages allow declaring variables without assigning an initial value or allocating memory without initializing it.
(Especially the latter is often important for efficiency.)

But any use of uninitialized variables or memory is almost certainly an error.
(The only flimsy exception arises when random behavior is intended, e.g., in a random number generator, and even then it is bad practice.)

\paragraph{Probable Typing Errors}
Sometimes the specification of a language makes certain expressions technically well-typed even though they are likely to be errors.

In languages with a universal type $U$, any equality $a==b$ is well-typed because $a:U$ and $b:U$.
However, $a==b$ is usually an error if the smallest common type of $a$ and $b$ is $U$.
For example, if $a:\Int$ and $b:List[\Int]$, then $a==b$ is a likely error (because it always returns $\false$).

In language with explicit type checks $\aisinst{a}{T}$ and type casts $\aasinst{a}{T}$, any such check or cast may be well-typed.
For example, if the language has $null$, indeed both must be well-typed because they succeed if $a==null$ independent of $T$.
But given $a:A$, such checks and casts are likely errors if $T$ is not a subtype of $A$.
For example, casting $a:List[\Int]$ into $Option[\Int]$ is a likely error (because it succeeds only when $a==null$, in which case it is a useless cast).

\paragraph{Null Checks}
In languages with $null$ pointers, it is often possible to analyze whether a pointer is dereferenced that may be null.

Given a value $a:A$, special cases include
\begin{compactitem}
 \item dereferencing $a$ if $a$ is known to be null
 \item dereferencing $a$ without checking $a==null$
 \item checking $a==null$ when $a$ has been previously dereferenced
\end{compactitem}

\paragraph{Unreachable Code}
If no execution can ever execute a certain command, we speack of unreachable code.
It is always a bug.

Many instances of unreachable code can be detected by systematic analysis.
This includes
\begin{compactitem}
  \item code in a branch of an if-else whose condition always has the same value
  \item code in a case distinction (switch) statement that come after a default case
  \item code after a return statement
\end{compactitem}
