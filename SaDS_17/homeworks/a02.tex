\documentclass[a4paper]{article}

\usepackage[course={Secure and Dependable Systems},number=2,date=2017-02-08,due=2017-03-02]{../../myhomeworks}

\newcounter{chapter} % needed for dependencies of mylecturenotes
\usepackage[root=../..]{../../mylecturenotes}
\usepackage{../../macros/algorithm}

\begin{document}

\header

\begin{problem}{Shellshock}{10}
Consider the shellshock example from the lecture notes.
In this problem, we implement a minimal shell that could exhibit the fault but will not because we design it well.

You can use any programming language.
However, it is best to use a programming language that supports good system design.
SML will work well; Java or C++ are OK. I personally recommend Scala.

\renewcommand{\bnf}[1]{{\color{red}#1}}
\begin{enumerate}
\item Implement datatypes for the following grammar, which represents the commands our shell can handle.
\[\begin{array}{lcl@{\tb}l}
 \text{commands} \\
 COMM & \bbc & fun\;NAME(NAME) \{COMM\} & \text{function definition} \\
      & \bnfalt & run\;NAME \rep{\bnfbracket{SPACE\; EXPR}} & \text{shell call}\\
      & \bnfalt & NAME(EXPR) & \text{function call}\\
      & \bnfalt & COMM ; COMM & \text{command sequence}\\
 \text{expressions} \\
 EXPR & \bbc & NAME & \text{variable} \\
      & \bnfalt& "\; \rep{\bnfbracket{\backslash\backslash \bnfalt \backslash" \bnfalt \bnfnegchoice{\backslash"}}} \;" & \text{string} \\
 \text{names} \\
 NAME & \bbc & \text{alphanumeric string} &
\end{array}\]
where red color indicates BNF meta-symbols.

You need one datatype per non-terminal with one constructor per production.
Let $Comm$ and $Expr$ be the types for $COMM$ and $EXPR$.

This language is pretty boring in order to be simple.
Feel free to add, e.g., if commands, number expressions, functions with return values etc.
However, watch out that writing the parser will become increasingly work-intensive.

\item Implement a parser for your data type. It should be of the form
\begin{acode}
\afun[Comm]{parseCommand}{command : String}{\ldots}\\
\afun[Expr]{parseExpr}{expr : String}{\ldots}
\end{acode}

\item Implement an interpreter for your data type. It should be of the form
\begin{acode}
\aclassA{Value}{}{}{}\\
\aclass{StringValue}{value: String}{Value}{}\\
\\
\aclassA{Def}{}{}{}\\
\aclass{ValDef}{name: String, value: Value}{Def}{}\\
\aclass{FunDef}{name: String, argName: String, body: Comm}{Def}{}\\\\
\afun[{List[Def]}]{interpret}{context: List[Def], command: Comm}{\ldots} \\
\afun[Value]{evaluate}{context: List[Def], expr: Expr}{\ldots}
\end{acode}
and interpret commands as follows:
\begin{compactitem}
\item $fun\;f(x)\{C\}$: return $FunDef(f,x,C)$
\item $run\;n\,arg_1\,\ldots\,arg_n$: call the shell function $n$ with the listed arguments
\item $f(e)$: evaluate $e$ to $v$, retrieve $FunDef(f,x,C)$ from the context, interpret $C$ with an additional $ValDef(x,v)$ in the context
\item $C ; D$: interpret $C$, then interpret $D$ with the $FunDef$s returned by $C$ added to the context
\end{compactitem}
and evaluate expressions as follows
\begin{compactitem}
\item $n$: retrieve $ValDef(n,v)$ from the context and return $v$
\item $"s"$: return the string $s$ with the escapes removed
\end{compactitem}

\item Implement main function that takes a string $s$, calls $c = parseComm(s)$, then calls $interpret(Nil,c)$.

Optionally, you can also
\begin{compactitem}
  \item read an entire file and parse+interpret every line in it
  \item read commands from standard input, at which point you have an actual shell
\end{compactitem}
Those steps are not required but recommended because they help with testing.

\item Now for the faulty functionality of bash, modify your program as follows:
 \begin{compactitem}
  \item At the beginning, try to parse every available environment variable that starts with $fun$ into a $Comm$.
  \item Whenever that succeeds, interpret the resulting commands, which returns $defs:List[Def]$. (*)
  \item Now call interpret on the input with $defs$ instead of $Nil$ as the initial context.
 \end{compactitem}

\item Activate the fault by showing that data in environment variables may lead to the execution of arbitrary shell commands.

\item Isolate the fault (This is how it was ``fixed'' in bash.) by considering only environment variables whose name begins with a certain prefix, e.g., $SHELL\_FUNC$.

\item Remove the fault by making sure that data is never executed.
To do so, change (*) so that it never calls $interpret$.
\end{enumerate}

For questions 1-5, submit a single program.
For questions 7-8, submit a single program.
Use comments as needed to understand which part solves which question.

For question 6, submit a screenshot from a normal shell session that demonstrates the failure of your shell.
\end{problem}

\end{document}
