\section{RSA}

The idea behind RSA is that if $N=p\cdot q$ for large prime numbers $p$ and $q$, it is very difficult to compute $p$ and $q$ from $n$.

\paragraph{Setup}
Choose two large primes $p$ and $q$ (typically of roughly equal size).
Put $N=p\cdot q$.

Now put $n=(p-1)(q-1)$. (Actually, any common multiple of the two numbers is fine.)
Note that $n=\phi(N)$.
Pick $e\in Z_n$ such that there is a $d\in\Z_n$ with $e\cdot d\Equiv_n 1$.
Such a $d$ exists if $\gcd(e,n)=1$ and is easy to compute (see Thm.~\ref{thm:math:extendedeuclid}).

The keys are defined as follows:
\begin{compactitem}
 \item public information (encryption key): $N$ and $e$
 \item private information (decryption key): $n$, $d$, $p$, and $q$
\end{compactitem}
Among the private information, only $N$ and $d$ are needed later on.
So $n$, $p$, and $q$ can be forgotten.
But they have to remain private---$p$ (or $q$) is enough to compute $n$ and $d$.

Different keys are often compared by their size.
That size is the number of bits in $N$.

\paragraph{Encryption}
Messages are numbers $x\in F_N$.
For example, we can choose the largest $k$ such that $2^k<N$ and use $k$-bit messages.

Encryption and decryption are functions $\Z_N\to \Z_N$ given by
\begin{compactitem}
 \item encryption: $x\mapsto x^e\modop N$
 \item decryption: $x\mapsto x^d\modop N$
\end{compactitem}

These are indeed inverse to each other:

\begin{theorem}
For all $x\in Z_N$, we have $(x^d)^e\Equiv_N (x^e)^d \Equiv_N x$.
\end{theorem}
\begin{proof}
In general, because $N=p\cdot q$ for prime numbers $p$ and $q$, we have that $x\Equiv_N y$ iff $x\Equiv_p y$ and $x\Equiv_q y$.

So we have to show that $x^{de}\Equiv_p x$.
(We also have to show the same result for $q$, but the proof is the same.)
We distinguish two cases:
\begin{compactitem}
\item $p|x$: Then trivially $x^{de}\Equiv_p x\Equiv_p 0$.
\item Otherwise. Then $p$ and $x$ are coprime.\\
   By construction of $e$ and $d$ and using Thm.~\ref{thm:math:extendedeuclid}, we have $k\in\N$ such that $e\cdot d+k\cdot n=1$.
   Thus, we have to show $x^{de}=x\cdot (x^{p-1})^{k\cdot(q-1)}\Equiv_p x$.
   That follows from $x^{p-1}\Equiv_p 1$ as known from Thm.~\ref{thm:math:fermatlittle}.
\end{compactitem}
\end{proof}

\paragraph{Attacks}
To break RSA, $d$ has to be computed.
There are $3$ natural ways to do that:
\begin{compactitem}
 \item Factor $N$ into $p$ and $q$. Then compute $d$ easily.
 \item Compute $n$ using $n=\phi(N)$ (which may be easier than finding $p$ and $q$). Then compute $d$ easily.
 \item Find $d$ such that $e\cdot d\Equiv_n 1$ (which may be easier than finding $n$).
\end{compactitem}
Currently these are believed to be equally hard.

It is believed that there is no algorithm for factoring $N$ that is polynomial in the number of bits of $N$.
That is no proved.
There are hypothetical machines (e.g., quantum computers) that can factor $N$ polynomially.

Note that checking if $N$ can be factored (without producing the factors) is polynomial, and practical algorithms exist (in particular, the AKS algorithms).
That is important to find the large prime number $p$ and $q$ efficiently.
\medskip

If there is indeed no polynomial algorithm, factoring relies on brute-force attacks that find all prime numbers $k<\sqrt{N}$ and test $k|N$.
Therefore, larger keys are harder than break to smaller ones.
Because of improving hardware, the key size that is considered secure grows over time.

Keys of size $1024$ are considered secure today, but because security is a relative term, keys of size $2048$ are often recommended.
Larger keys are especially important if data is needed to remain secure far into the future, when faster hardware will be available.