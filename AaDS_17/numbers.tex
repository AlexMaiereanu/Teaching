\section{Specification}

The sets $\N$, $\Z$, and $\Q$ are well-known from mathematics.

We cannot implement data structures for $\R$ and $\C$ because they are uncountable.
We can at most work with $\Q+\Q i$, which is the set of complex numbers whose real and imaginary parts are rational.

\section{Implementation}

Working with $\Z$ as opposed to $\Z_m$ for some $m$ is called \emph{arbitrary precision arithmetic}.
In some programming languages, $\Int$ is already $\Z$.
In other language, $\Int$ is $\Z_m$ for some $m$---in those languages, there is usually a library that defines $\Z$.

A data structure for $\Q$ can be defined by using pairs of integers.

We usually do not use a special data structure for $\N$ and instead just use the positive values of $\Z$.
Alternatively, we can give a (very inefficient) definition of $\N$ as an inductive type as in Ex.~\ref{ex:ad:euclid2}.