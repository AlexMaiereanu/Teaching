\section{Specification}

% (lattice operations) contains/insert/delete for mutable sets

The set $Set(A)$ contains the finite subsets of $A$.
It is countable if $A$.

Sets can be mutable or immutable.
The main operations for immutable sets are:

\begin{ctabular}{|l|l|l|}
\hline
function & returns & effect \\
\hline
$contains(x\in Set[A], a\in A)\in \Bool$ & $\true$ iff $a\in x$ & none \\
$insert(x\in Set[A], a\in A)\in Set[A]$ & $x\cup\{a\}$ & none \\
$delete(x\in Set[A], a\in A)\in Set[A]$ & $x\sm\{a\}$ & none \\
\hline
\end{ctabular}

The main operations for mutable sets are:

\begin{ctabular}{|l|l|l|}
\hline
function & returns & effect \\
\hline
$contains(x\in Set[A], a\in A)\in \Bool$ & $\true$ iff $a\in x$ & none \\
$insert(x\in Set[A], a\in A)\in\Unit$ & nothing & $x:=x\cup\{a\}$ \\
$delete(x\in Set[A], a\in A)\in\Unit$ & nothing & $x:=x\sm\{a\}$ \\
\hline
\end{ctabular}

\section{List-Based Sets}

The easiest data structure for $Set[A]$ is $ListSet[A]$.
It stores a set as a list without repetitions.

The operations on $ListSet[A]$ are defined in the same way as for $List[A]$ with one exception: the $insert(x,a)$ operation does nothing if $x$ already contains $a$.
\medskip

If $n$ is the size of the $ListSet$, $contains$, $insert$, and $delete$ takes $\Theta(n)$ on average.
That can quickly be a problem.
For example, building a set with $n$ elements step-by-step requires $n$ insertion and thus costs $\Theta(n^2)$.

The data structures below are more sophisticated in order to beat the complexity of $ListSet$.

\section{Hash Sets}

A hash function for the set $A$ is just a function $hash:A\to \Z_m$.
A collision is a pair $x,y\in A$ such that $hash(x)=hash(y)$.

A good hash function should be fast and rarely have collisions.
An (unrealistically) ideal hash function runs in $O(1)$ and the probability of $hash(x)=hash(y)$ is $1/m$.
Those two properties work against each other: For example, it is easy to be fast by always returning $0$, but that has maximally many collisions.
Vice versa, it is easy to minimize collisions by choosing $h$ carefully, but then $hash$ may be very expensive to compute.
Thus, hash functions must make a trade-off.

For a fixed hash function $hash:A\to \Z_m$, the data structure $HashSet[A]$ is given by
\begin{acode}
\atypedef{HashSet[A]}{Array[ListSet[A]](m)}\\
\afunI{insert}{h: HashSet[A], a:A}{insert(h[hash(a)],a)}\\
\afunI{delete}{h: HashSet[A], a:A}{delete(h[hash(a)],a)}
\end{acode}

If $n$ is the size of the set, the sets $h[0]$, \ldots, $h[m-1]$ have average size $n/m$.
Thus, $insert$ and $delete$ take $n/m$ on average.
\medskip

If $m$ is too big, most of the sets $h[i]$ will be empty, which wastes memory.
Optimized data structures for hash sets can dynamically increase $m$ if $n$ increases to avoid waste.

\section{Binary Search Trees}

If we have a total order $\leq$ on $A$, we can use ordered trees to realize $Set[A]$.

\section{Red-Black Trees}

% red-black trees (BST with log n guarantee for insert/delete)