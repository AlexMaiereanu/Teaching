\section{Specification}

The set $Set[A]$ contains the finite subsets of $A$.
It is countable if $A$.

Sets can be mutable or immutable.
The main operations for immutable sets are:

\begin{ctabular}{|l|l|l|}
\hline
function & returns & effect \\
\hline
$contains(x\in Set[A], a\in A)\in \Bool$ & $\true$ iff $a\in x$ & none \\
$insert(x\in Set[A], a\in A)\in Set[A]$ & $x\cup\{a\}$ & none \\
$delete(x\in Set[A], a\in A)\in Set[A]$ & $x\sm\{a\}$ & none \\
\hline
\end{ctabular}

% lattice operations

The main operations for mutable sets are:

\begin{ctabular}{|l|l|l|}
\hline
function & returns & effect \\
\hline
$contains(x\in Set[A], a\in A)\in \Bool$ & $\true$ iff $a\in x$ & none \\
$insert(x\in Set[A], a\in A)\in\Unit$ & nothing & $x:=x\cup\{a\}$ \\
$delete(x\in Set[A], a\in A)\in\Unit$ & nothing & $x:=x\sm\{a\}$ \\
\hline
\end{ctabular}

In both cases, we often need operations for combining and comparing sets:
\begin{ctabular}{|l|l|l|}
\hline
function & returns & effect \\
\hline
$equal(x\in Set[A], y\in Set[A])\in \Bool$ & $\true$ iff $x=y$ & none \\
$union(x\in Set[A], y\in Set[A])\in Set[A]$ & $x\cup y$ & none \\
$inter(x\in Set[A], y\in Set[A])\in Set[A]$ & $x\cap y$ & none \\
$diff(x\in Set[A], y\in Set[A])\in Set[A]$ & $x\sm y$ & none \\
\hline
\end{ctabular}

Equality is listed explicitly here because it can be very complex.
For most data structures such as the ones for lists and trees, equality is straightforward.
This may or may not be the case for data structures for sets.

\section{Data Structures}

\subsection{Using Bit Vectors}\label{sec:ad:vectorset}

If $A$ is finite with $|A|=m$, an easy data structure for $Set[A]$ are bit vectors of length $m$ such as $Array[\Bool](m)$.
Given such a vector $a$, we put $a[i]=\true$ to represent that $i$ is in the set.

Then we can implement $insert$ and $delete$ easily in $\Theta(1)$.
We can also implement $equal$, $union$, $inter$, and $diff$ easily in $\Theta(m)$.

A major drawback is the memory requirement: We need $\Theta(m)$ for each $x:Set[A]$, which is only feasible for small $m$.

\subsection{Using Lists}\label{sec:ad:listset}

For large or infinite $A$, a better data structure for $Set[A]$ is $ListSet[A]$.
It represents the set $\{a_1,\ldots,a_n\}$ as the list $[a_1,\ldots,a_n]$.
Thus, it represent sets as lists without repetition.

The operations on $ListSet[A]$ are defined in the same way as for $List[A]$ with one exception: the $insert(x,a)$ operation does nothing if $x$ already contains $a$.
\medskip

If $n$ is the size of the $ListSet$, the operations $contains$, $insert$, and $delete$ takes $\Theta(n)$.
However, higher-level operations like building a set with $n$ elements  step-by-step by calling $insert$ $n$ times requires $n$ insertions and thus costs $\Theta(n^2)$.

Moreover, these operations require calls to the equality on $A$.
For example, to implement $contains(x,a)$, we have to compare $a$ to every element of $x$.
That may be easy, e.g., if $A=\Int$.
But it can be arbitrarily costly if $A$ is more complex data structure itself.

For equality, union, intersection, and difference of $x$ and $y$, we may have to compare every element of $x$ with every element of $y$.
So it may take $O(|x|\cdot|y|)$.

These operations quickly become too costly for large subsets of $A$.

\subsection{Hash Sets}\label{sec:ad:hashset}

Hash sets try to comine the advantages of bit vector and list sets.
The key parameter is a function $hash:A\to \Z_m$.
This is called the hash function.

$hash$ has two purposes:
\begin{compactitem}
 \item The set $A$ is supported by a finite, managably small set $\Z_m$.
   That makes if feasible to use arrays of length $m$.
 \item The equality operation on $A$ is supported by the $O(1)$ equality on $\Z_m$.
   To check $a=a'$, we first check $hash(a)=hash(a')$.
   If false, we know $a\neq a'$; otherwise, we call the usual equality on $A$.
   That minimizes the number of equality on $A$ is called.
\end{compactitem}

Of course, the function $hash$ will usually not be injective.
A \textbf{collision} is a pair $x,y\in A$ such that $hash(x)=hash(y)$.

A good hash function should be fast and rarely have collisions.
An (unrealistically) ideal hash function runs in $O(1)$ and the probability of $hash(x)=hash(y)$ is $1/m$.
Those two properties work against each other: For example, it is easy to be fast by always returning $0$, but that has maximally many collisions.
Vice versa, it is easy to minimize collisions by choosing $hash$ carefully, but then $hash$ may be very expensive to compute.
Thus, hash functions must make a trade-off.

For a fixed hash function $hash:A\to \Z_m$, the data structure $HashSet[A]$ is given by
\begin{acode}
\atypedef{HashSet[A]}{Array[ListSet[A]](m)}\\
\afunI{insert}{h: HashSet[A], a:A}{insert(h[hash(a)],a)}\\
\afunI{delete}{h: HashSet[A], a:A}{delete(h[hash(a)],a)}
\end{acode}
\medskip

If $n$ is the size of the subset of $A$, the sets $h[0]$, \ldots, $h[m-1]$ have average size $n/m$.
Thus, $contains$, $insert$ and $delete$ take $n/m$ on average.
$equal$, $union$, $inter$, and $diff$ are similarly sped up.
\medskip

Asymptotically, hash sets do not beat list sets because they only spped up by a constant factor.
But that constant factor is a critical improvement.

The speed up is bigger if $m$ is bigger.
However, the memory requirement increases linearly with $m$: Even the empty subset requires $\Theta(m)$ space and $\Theta(m)$ time to initialize that space.

Optimized data structures for hash sets can dynamically choose $m$ in order to find a good trade-off.
Often users of the $HashSet$ data structure can choose the value of $m$.
That can help if they know in advance how big the subset is going to get and what kind of operations will be called.

\subsection{Red-Black Trees}\label{sec:ad:redblacktree}

% red-black trees (BST with log n guarantee for insert/delete)

\subsection{Binary Search Trees}\label{sec:ad:bst}

% If we have a total order $\leq$ on $A$, we can use ordered trees to realize $Set[A]$.
