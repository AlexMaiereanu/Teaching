\section{Exponentiation}\label{sec:ad:exp}

\subsection{Specification}\label{sec:ad:exp:spec}

The function $power(x\in\Z,n\in\N)\in\N$ (also written as $x^n$) returns the $n$-the power of $x$ defined by
\[x^0=1\]
\[x^{n}=x\cdot x^{n-1} \tb\mif n > 0\]

By induction on $n$, we show this indeed specifies a unique function.

\subsection{Naive Algorithm}\label{sec:ad:exp:naive}

It is straightforward to give an algorithm for exponentiation.
For example,
\begin{acode}
\afun[\N]{power}{x:\Z,n:\N}{
  \aifelse{x==0}{1}{x\cdot power(x,n-1)}
}
\end{acode}

\paragraph{Correctness}
The correctness of this algorithm is immediate because it follows the specification literally.
For example, $T(x,n)=n$ is already a termination ordering.

\paragraph{Complexity}
Assuming that all multiplications take $O(1)$ no matter how big $x$ is, the complexity of this algorithm is $\Theta(n)$ because we need $n$ multiplications and recursive calls.

\subsection{Square-and-Multiply Algorithm}\label{sec:ad:exp:sqmult}

It is easy to think that $\Theta(n)$ is also the complexity of the specification, i.e., that there is no sub-linear algorithm for it.
But that is not true.

Consider the square-and-multiply algorithm:
\begin{acode}
\afun[\N]{sqmult}{x:\Z,n:\N}{
  \aifelse{n==0}{1}{
	  r := sqmult(x, n\divop 2)\\
	  sq := r\cdot r\\
	  \aifelseI{n \modop 2==0}{sq}{x\cdot sq}
  }
}
\end{acode}

\paragraph{Correctness}
To prove the correctness of this algorithm, we note that
\[x^{2i+0}=(x^i)^2\]
\[x^{2i+1}=x\cdot(x^i)^2\]
Moreover, we know that $n=2(n\divop 2)+(n\modop 2)$.
Partial correctness of $sqmult$ follows immediately.

To prove termination, we observe that $T(x,n)=n$ is a termination ordering: $n\divop 2$ always decreases (because $n\neq 0$) and remains positive.

\paragraph{Complexity}
Computing the run time of a recursive function often leads to a recurrence relation: The function occurs on both sides with different arguments.
In this case, we get:
 \[C(n)=C(n\divop 2) + c\]
where $c\in O(1)$ is the constant-time effort needed in each iteration.
We systematically expand this further
 \[C(n)=C(n\divop 2) + c = C(n\divop 2\divop 2) + 2\cdot c=\ldots=C(n\overbrace{\divop 2\ldots\divop 2}^{k+1\,\text{times}})+(k+1)\cdot c\]

Now let $n=(b_k\,\ldots\,b_0)_2$ be the binary representation of the exponent.
We know that $k=\lfloor\log_2 n\rfloor$ and $n\overbrace{\divop 2\ldots\divop 2}^{k+1\,\text{times}}=0$.
Moreover, we know from the base case that $C(0)=1$.

Substituting these above yield
\[C(n)\in O(1)+\Theta(\log_2 n)\cdot O(1)=\Theta(\log_2 n)\]

Thus, we can compute power in logarithmic time.

\section{Fibonacci Numbers}\label{sec:ad:fib}

\subsection{Specification}\label{sec:ad:fib:spec}

The Fibonacci numbers $Fib(n\in\N)\in\N$ are defined by
\[fib(0)=0\]
\[fib(1)=1\]
\[fib(n)=fib(n-1)+fib(n-2) \tb\mif n>1\]

By induction on $n$, we prove that this indeed specifies a unique function.

Moreover, we can prove the non-obvious result that
 \[fib(n)=\frac{\phi^n-(1-\phi)^n}{\sqrt{5}} \tb\mfor \phi=\frac{1+\sqrt{5}}{2}\]
($\phi$ is also called the golden ratio.)
That can be further simplified to
 \[fib(n)=round\left(\frac{\phi^n}{\sqrt{5}}\right)\]
where we round to the nearest integer.

\subsection{Naive Algorithm}\label{sec:ad:fib:naive}

It is straightforward to give an algorithm for computing Fibonacci numbers.
For example:
\begin{acode}
\afun[\N]{fib}{n:\N}{
  \aifelse{n\leq 1}{n}{fib(n-1)+fib(n-2)}
}
\end{acode}

\paragraph{Correctness}
The correctness of this algorithm is immediate because it follows the specification literally.
For example, $T(n)=n$ is a termination ordering.

\paragraph{Complexity}
We obtain the recurrence relation $C(n)=C(n-1)+C(n-2)+c$ where $c\in O(1)$ is the constant-time effort of the recursion.
That is the same recurrence as for the definition of the Fibonacci numbers themselves, thus $C(n)\in O(fib(n))=\Exp$.

This naive approach is exponential because every functions spawns $2$ further calls.
Each time $n$ is reduced only by $1$ or $2$, so we have to double the number of calls about $n$ times to $\Theta(2^n)$ calls.

\subsection{Linear Algorithm}\label{sec:ad:fib:linear}

It is straightforward to improve on the naive algorithm turning an exponential into a liner solution.
For example:
\begin{acode}
\afun[\N]{fib}{n:\N}{
  \aifelse{n\leq 1}{n}{
    prev := 0\\
    current := 1 \\
    i = 1 \\
    \awhile{i<n}{
      next := current + prev \\
      prev := current \\
      current := next\\
      i := i+1
    }\\
    \areturn{current}
   }
}
\end{acode}

\paragraph{Correctness}
As a loop invariant, we can use
\[F(n,prev,current,i) = prev==fib(i-1)\wedge current==fib(i)\]
which is straightforward to verify.
After the loop, we have $i==n$ and thus $current=fib(n)$, which yields partial correctness.

As a termination ordering, we can use $T(n,prev,current,i)=n-i$.
Again this is straightforward to verify.

\paragraph{Complexity}
Both the code before and inside the loop take $O(1)$, and the loop is repeated $n-1$ times.
Thus, the complexity is $O(n)$.

\subsection{Inexact Algorithm}\label{sec:ad:fib:inexact}

It is tempting to compute $fib(n)$ directly using $fib(n)=round(\phi^n/\sqrt{5})$.
Because we can precompute $1/\sqrt{5}$, that requires $n+1$ floating point multiplications, i.e., also $O(n)$.

However, it is next to impossible verify the correctness of the algorithm.
While termination is trivial, partial correctness does not hold.
We know that the formula $fib(n)=round(\phi^n/\sqrt{5})$ is true, but that has no immediate use for floating point arithmetic.
Rounding errors will accumulate over time and may eventually lead to a false result.

\subsection{Sublinear Algorithm}\label{sec:ad:fib:sublinear}

Maybe surprisingly, we can still do better.
Inspecting the body of the while loop in the linear algorithm, we see that we can rewrite the assignments as
\[(current,prev):=(current+prev, current)\]
which we can write in matrix form as
\[(current,prev):=(current,prev)\cdot\begin{pmatrix}1&1\\1&0\end{pmatrix}\]

Thus, we obtain
\[(fib(n),fib(n-1))= (1,0)\cdot\begin{pmatrix}1&1\\1&0\end{pmatrix}^n \tb\mfor n>0\]

We can now pick any algorithm for computing the $n$-power of a matrix, e.g., by using square-and-multiply from Sect.~\ref{sec:ad:exp:sqmult} for matrices.

\paragraph{Correctness}
Correctness follows from the correctness of square-and-multiply.

\paragraph{Complexity}
Square-and-multiply has complexity $O(\log n)$.
Thus, we can compute $fib(n)$ with logarithmic complexity.

\section{Matrices}

\subsection{Specification}

We write $\Z^{mn}$ for the set $(\Z^n)^m$ of vectors over vectors (i.e., matrices) over integers.

We define two operations on matrices:
\begin{compactitem}
\item Addition: For of $x,y\in \Z^{mn}$, we define $x+y\in\Z^{mn}$ by
\[(x+y)_{ij}=x_{ij}+y_{ij}\]

\item Multiplication: For $x\in \Z^{lm}$ and $y\in \Z^{mn}$, we define $x\cdot y\in\Z^{ln}$ by
\[(x\cdot y)_{ij}=x_{i1}\cdot y_{1j} +\ldots +x_{im}\cdot y_{mj}\]
\end{compactitem}


\subsection{Naive Algorithms}

Vectors and matrices are best stored using arrays.
We assume that
\begin{compactitem}
 \item $Mat$ is the data structure of arrays of arrays of the same length of integers,
 \item if $x$ is an object of $Mat$, then $x.rows$ is the length of the array and $x.columns$ is the length of the inner arrays,
 \item $\anew{Mat}{m,n}$ produces a new array of length $m$ of arrays of length $n$ in which all fields are initialized as $0$.
\end{compactitem}

Then we have the straightforward algorithms
\begin{acode}
\afun[Mat]{add}{x:Mat,y:Mat}{
  r = \anew{Mat}{x.rows,x.columns}\\
  \afor{i}{1}{x.rows}{
    \afor{j}{1}{x.columns}{
      r.i.j := x.i.j+y.i.j
     }
  }\\
  \areturn{r}
} \\
\\
\afun[Mat]{mult}{x:Mat,y:Mat}{
  r = \anew{Mat}{x.rows,y.columns}\\
  \afor{i}{1}{x.rows}{
    \afor{j}{1}{y.columns}{
      \afor{k}{1}{x.columns}{
        r.i.j := r.i.j + x.i.k\cdot y.k.j
      }
    }
  }\\
  \areturn{r}
}
\end{acode}

\paragraph{Correctness}
The algorithms directly implement the definitions.
Thus, correctness---seemingly---obvious.

But there is one subtlety: The functions take two arbitrary matrices---there is no way to force the user to pass matrices of the correct dimensions.
Therefore, we have to state correctness a bit more carefully:
\begin{compactitem}
 \item for $z:=add(x,y)$
   \begin{compactitem}
     \item[precondition:] $x.rows==y.rows$ and $x.columns==y.columns$,
     \item[postcondition:] $z==x+y$ and $z.rows==x.rows$ and $z.columns==x.columns$.
   \end{compactitem}
 \item for $z:=mult(x,y)$
   \begin{compactitem}
     \item[precondition:]  $x.rows==y.columns$
     \item[postcondition:] $z:=mult(x,y)$ is $x\cdot y$ and $z.rows==x.rows$ and $z.columns==y.columns$
   \end{compactitem}
\end{compactitem}
Then we can easily show that $add$ and $mult$ are correct in the sense that the precondition implies the postcondition.

\paragraph{Complexity}
Assuming that all additions and multiplications take constant time, the complexity is easy to analyze.
For addition it is $\Theta(mn)$ and for multiplication $\Theta(lmn)$ where $l$, $m$, and $n$ are the dimensions of the respective matrices.

For addition, we can immediately see that we cannot improve on $\Theta(mn)$: Just creating the new array and returning it already takes $\Theta(mn)$ steps.
Thus, $\Theta(mn)$ is the complexity of the specification, and the naive algorithm is optimal.

This is not obvious for multiplication.
Using the same argument, we can say that the complexity of multiplication is $\Omega(ln)$.
But there cannot be an $\Theta(ln)$-algorithm because $m$ must matter---if $m$ increases, it must take longer.

\subsection{Strassen's Multiplication Algorithm}

Inspecting the definition of matrix multiplication, we see that we can split up matrices into rectangular areas of submatrices, for example, like so:
\[\begin{pmatrix}x_{11} & x_{12} & x_{13} & x_{14} \\ x_{21} & x_{22} & x_{23} & x_{24} \\ x_{31} & x_{32} & x_{33} & x_{34} \\ x_{41} & x_{42} & x_{43} & x_{44}\end{pmatrix}
= \begin{pmatrix}
    \begin{pmatrix}x_{11} & x_{12}\\ x_{21} & x_{22}\end{pmatrix} & \begin{pmatrix} x_{13} & x_{14} \\ x_{23} & x_{24} \end{pmatrix} \\
    \begin{pmatrix}x_{31} & x_{32}\\ x_{41} & x_{42}\end{pmatrix} & \begin{pmatrix} x_{33} & x_{34} \\ x_{43} & x_{44} \end{pmatrix}
  \end{pmatrix}
\]
Moreover, if matrices are split up like that, we can still obtain their product in the same way using recursive matrix multiplication:
\[\begin{pmatrix} a & b \\ c & d\end{pmatrix}\cdot \begin{pmatrix} e & f \\ g & h\end{pmatrix}=
  \begin{pmatrix} ae+bg & af+bh \\ ce+dg & cf+dh\end{pmatrix}=\begin{pmatrix} p & q \\ r & s\end{pmatrix}\]

Strassen's algorithm works in the general.
But for simplicity, we only consider the case $l=m=n$, i.e., we are multiplying square matrices.
Then the naive algorithm has complexity $\Theta(n^3)$, and we know the specification has complexity $\Omega(n^2)$.
The question is to find a solution in between.

We further simplify to $n=2^k$, i.e., we can recursively subdivide our $2^k$-matrices $4$ $2^{k-1}$-matrices.
Then we can design a recursive algorithm that only needs $k$ nested recursions.

The complexity depends on the details of the implementation.
Naively, computing $p,q,r,s$ requires $8$ recursive calls to multiplications and $4$ additions of $2^{k-1}$-matrices.
That yields
 \[C(n)=8\cdot C(n/2) + \Theta(n^2) = \ldots = 8^k\cdot C(1)+\Theta(n^2)\]
Because $k=\log_2 n$ and $C(1)\in O(1)$, that yields $C(n)\in\Theta(n^{\log_2 8})=\Theta(n^3)$.

However, Strassen observed that we can do better.
With some fiddling around, we can replace the $8$ multiplications and $4$ additions with $7$ multiplications and $18$ additions.
The extra additions do not harm because they are $\Theta(n^2)$.
But turning the $8$ into a $7$ yields $C(n)=\Theta(n^{\log_2 7})$.
Thus, Strassen's algorithm reduces $n^3$ to $n^{2.81\ldots}$, which can yield practically relevant improvements for relatively small $n$, e.g., $n\approx 30$.
\medskip

Even more efficient algorithms are found regularly.
The current record is $\Theta(n^{2.37\ldots})$.
However, the sufficiently large $n$ for which these are actually faster than Strassen's algorithm is so large that they have no practical relevance at the moment.

% multiplication: polynomials, n-digit numbers (optimal solution unknown)