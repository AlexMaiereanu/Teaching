\documentclass[a4paper]{article}

\usepackage[course={Algorithms and Data Structures},number=3,date=2017-02-21,due=2017-03-02,unpublished]{../../myhomeworks}

\newcounter{chapter} % needed for dependencies of mylecturenotes
\usepackage[root=../..]{../../mylecturenotes}
\usepackage{../../macros/algorithm}

\begin{document}

\header

\begin{problem}{Problem Complexity}{6+6}
What is the complexity class of the worst-case time complexity $C(n)$ of the following \emph{problems}:
\begin{enumerate}
\item Find the smallest element in a list of length $n$ of integers.
\item Find an unknown $x\in \N$ between $0$ and $n$ by repeatedly asking yes/no questions about $x$.\\
(Any question is allowed that has a well-defined answer. For example, you may ask ``Is $x$ prime?'' or ``Is $x==5$?'' A solution must determine $x$ with perfect certainty in all cases.)
\end{enumerate}

In each case, answer the question in two parts:
\begin{compactitem}
\item Give an algorithm (in pseudo-code or a programming language) that is in the given complexity class. (3 points each)
\item Argue informally (but convincingly) why there can be no better algorithm. (3 points each)
\end{compactitem}
\end{problem}

\begin{problem}{Complexity Analysis}{5}
The following is a (highly simplified variant of an) example that came up last year in the author's research:
Consider the following function in the Scala programming language:

\begin{lstlisting}
def processString(s: String) {
  var rest: String = s
  while (s != "") {
    if (rest.startsWith("foo")) {
      // does not matter, assume this takes O(1)
    } else {
      // does not matter, assume this takes O(1)
    }
    rest = s.substring(1)
  }
}
\end{lstlisting}
Here \lstinline|startsWith| and \lstinline|substring| are methods on strings from the Java library (which Scala can call with only constant-time overhead).

Let $C(n)$ be the run time of this function where $n$ is the length of the input string.
What is the complexity class of $C(n)$ and why?

You can try the program yourself on increasingly large input to find out. (If you prefer, you can use Java instead of Scala---the effect is the same.)
In the author's case, the surprising effect was noticed by a student trying to process a $100$ MB text file, e.g., when $n>10^8$. The problem is already noticeable for smaller values of $n$.
\end{problem}

\begin{problem}{Polynomial Algorithms}{$\infty$}
Consider the following problem: Given a natural number that uses $n$ bits, find a non-trivial factor of $n$.

Here ``non-trivial factor'' means a number $p|n$ such that $1<p<n$.

Give an algorithm with polynomial run-time complexity $C(n)$ or show that no such algorithm exists.

\begin{hint}
There is no typo in this problem.
\end{hint}
\end{problem}

\end{document}
