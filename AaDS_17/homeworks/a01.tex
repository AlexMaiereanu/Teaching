\documentclass[a4paper]{article}

\usepackage[course={Algorithms and Data Structures},number=1,date=2017-02-07,duedate=2017-02-16]{../../myhomeworks}

\newcounter{chapter} % needed for dependencies of mylecturenotes
\usepackage[root=../..]{../../mylecturenotes}
\usepackage{../../macros/algorithm}

\begin{document}

\header

% exercise: rational numbers, multiplication, division, normalization through gcd

\begin{problem}{Euclidean Algorithm}{4+4+4}
Implement the Euclidean algorithm in multiple programming languages.

Specifically, pick one language from \textbf{each} of the following groups:
\begin{compactitem}
 \item untyped languages: Python, Javascript, PHP
 \item typed languages favoring imperative programming: Java, C++
 \item typed languages favoring functional programming: SML, Scala, Haskell
\end{compactitem}

Your implementations should work with \emph{arbitrary} natural numbers, i.e., numbers of unlimited size.
The technical term for that is \emph{arbitrary precision arithmetic}.
All programming languages provide that in one way or another.
But the built-in type $\Int$ may or may not offer arbitrary precision---if it does not, large numbers such as $35!$ will cause overflow errors.
See \url{https://en.wikipedia.org/wiki/List_of_arbitrary-precision_arithmetic_software} for an overview of how to do arbitrary precision arithmetic in different languages.

\begin{hint}
If you do not want to install multiple programming languages, you can use a number of websites to program online.
Just google \verb|online ide| to find several good sites.
\end{hint}
\end{problem}

\begin{problem}{Rational Numbers}{5+5+5}
The data structure of rational numbers consists of the following:
\begin{compactitem}
  \item a rational number is a pair of an integer $d$ and a positive natural number $e$ (both using arbitrary precision)
  \item two rational numbers $(d,e)$ and $(d',e')$ are equal iff $d\cdot e'=e\cdot d'$
  \item the normal form of a rational number $(d,e)$ is $(d/g, e/g)$ where $g=\gcd(d,e)$ \\
\end{compactitem}

For two rational numbers, we can define several operations
\begin{compactitem}
 \item addition: $(d,e)+(d',e'):=(d\cdot e'+e\cdot d', e\cdot e')$
 \item additive inverse (negative): $-(d,e):=(-d,e)$
 \item multiplication: $(d,e)+(d',e'):=(d\cdot d', e\cdot e')$
 \item multiplicative inverse: $1/(d,e):=(e,d)$
\end{compactitem}

Implement this data structure and all mentioned operations in \textbf{the same three} programming languages as above.
Use the Euclidean algorithm to compute the normal form.

\begin{hint}
If you do not understand the definitions, work out some examples. It should become clear quickly.
\end{hint}
\end{problem}

\end{document}
