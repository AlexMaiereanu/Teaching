\documentclass[a4paper]{article}

\usepackage[course={Algorithms and Data Structures},number=5,date=2017-03-07,due={2017-03-16, 11:00},unpublished]{../../myhomeworks}

\newcounter{chapter} % needed for dependencies of mylecturenotes
\usepackage[root=../..]{../../mylecturenotes}
\usepackage{../../macros/algorithm}

\begin{document}

\header

\begin{problem}{Sorting Lists of Integers}{8+8}
We want to sort lists of integers according to $\leq$.

Implement $2$ out of bubble sort, insertion sort, merge sort, and quick sort in any programming language(s).
\medskip

Remark: Alternatively, you may implement the algorithms for arbitrary arguments $ord:TotOrd[A], x:List[A]$.
(That's actually the same amount of work once you have understood polymorphism.)
\end{problem}

\begin{problem}{Sorting List}{8}
Adapt one of your two sorting implementations from the previous question to sort lists of strings lexicographically.
\medskip

Remark: If you gave a polymorphic solution above, these points are for free using $Lexicographic:TotOrd[\String]$ from before.
Otherwise, you have to duplicate your work.
\end{problem}

\begin{problem}{Comparing Sorting Algorithms}{5}
Using the lecture notes, literature, the internet etc., make a table that compares as many sorting algorithms as you can find.
For each algorithm, collect at least
\begin{compactitem}
 \item $\Theta$-class of the best-case, average-case, and worst-case time complexity (measured in the length of the list)
 \item whether it can be used as an in-place algorithm
\end{compactitem}
\medskip

Remark: Such questions often occur in job interviews or in standardized tests, e.g., for grad school admission.
So it is good to have such a table handy.
\end{problem}

\end{document}
