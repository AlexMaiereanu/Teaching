The specification and several data structures for mutable and immutable lists are already discussed in Sect.~\ref{sec:ad:listsort:spec}.

Here we only discuss some additional data structures for the set $A^*$.

\section{Stacks}\label{sec:ad:stack}

$Stack[A]$ is a data structure for the set $A^*$.

$Stack[A]$ is very similar to $List[A]$.
The difference is that $Stack[A]$ provides \emph{less} functionality.
While $List[A]$ is a general-purpose list, $Stack[A]$ is custom-fitted to one specific, very common use case.
By requiring fewer operations, they allow more optimized implementations.

Stacks can be mutable or (less commonly) immutable.
Here we will use the mutable variant.
The functions for mutable stacks are:

\begin{ctabular}{|l|l|l|}
\hline
function & returns & effect \\
\hline
$push(x\in A^*, a\in A)\in\Unit$ & nothing & prepend $x$ to $A$\\
$pop(x\in A^*)\in A^?$ & the first element of $x$ (if any) & remove the first element of $x$ \\
$top(x\in A^*)\in A^?$ & the first element of $x$ (if any) & none \\
\hline
\end{ctabular}

The intuition behind stacks is that they provide a LIFO store of data.
LIFO means last-in-first-out because every $pop$ returns the most recently pushed value.
This is exactly the behavior of a literal stack of items: We can put an item on top of a stack ($push$), remove an item from the stack ($pop$), or check what item is on top ($top$).
We cannot easily see or remove the other items.

Very often, the LIFO behavior is exactly what is needed.
For example, when we solve a maze, we can push every decision we make.
When we hit a dead end, we trace back our steps---for that, we have to pop the most recent decision, and so on.

\section{Queues}\label{sec:ad:queue}

Queues are very similar to stacks.
Everything about stacks also applies to queues except for the following.

The functions for mutable queues are:

\begin{ctabular}{|l|l|l|}
\hline
function & returns & effect \\
\hline
$enqueue(x\in A^*, a\in A)\in\Unit$ & nothing & append $x$ to $A$\\
$dequeue(x\in A^*)\in A^?$ & the first element of $x$ & remove the first element of $x$ \\
$empty(x\in A^*)\in\Bool$ & true if $x$ is empty & none \\
\hline
\end{ctabular}

The intuition behind queues is that they provide a FIFO store of data.
FIFO means first-in-first-out because every $dequeue$ returns the least recently enqueued value.
This is exactly the behavior of a literal queue of people: Every newcomer has to queue up at the end of the queue ($enqueue$), and every time a server is ready the first in line gets served ($dequeue$).
Newcomers cannot cut in line, and the server cannot easily see who else is waiting.

Very often, the FIFO behavior is exactly what is needed.
For example, when we have a list of tasks that need to be done.
Every time we create a new task, we enqueue it, and whenever we have time we dequeue the next task.

Queues are often used when components exchange messages or commands.
In that case, some components---called the producers---only call enqueue, and other components---called the consumers---only call dequeue.
For example, the producers can be different programs, $A$ is the type of print jobs, and the consumers are different printers.

More complex queue data structures may also for dequeueing based on priority (see also Sect.~\ref{sec:ad:heapqueue}).

\section{Buffers}\label{sec:ad:buffer}

Buffers are very similar to queues.
$Buffer[A]$ is usually optimized for enqueueing and dequeueing many elements of $A$ at once.
While stacks and queues can be easily implemented using linked lists, buffers usually use array to be faster.

For example, buffers are used when a program is writing to a file.
In that case, a $Buffer[\Char]$ is used holding the characters that are written to the file.
The write command does not actually write strings to the file---it only enqueues them in the buffer.
That is advantageous because enqueueing to a buffer in memory is much faster than writing to the hard drive.
While the program is already moving on, the hard drive is still busy dequeueing as fast as it can and writing all characters to the file.

\section{Iterators}\label{sec:ad:iter}

\subsection{Specification}

$Iterator[A]$ is a data structure for the set $A^*$.

Iterators are usually mutable.
Their functionality is even more restricted than the one of stacks and queues:

\begin{ctabular}{|l|l|l|}
\hline
function & returns & effect \\
\hline
$getNext(x\in A^*)\in A$ & the first element of $x$ & remove the first element of $x$ \\
$hasNext(x\in A^*)\in\Bool$ & $\true$ if $x$ is not empty & none \\
\hline
\end{ctabular}

The typical way to use an iterator $i\in Iterator[A]$ is the following:
\begin{acode}
\awhile{hasNext(i)}{
  a := getNext(i)\\
  \text{do something with $a$ here}\\
}
\end{acode}
This is called \textbf{traversing} the iterator.
Afterwards the iterator is traversed and cannot be used again.

$Iterator[A]$ may look somewhat boring.
In order to understand the value of iterator, we have to make one definition:
A data structure $D[A]$ is called \textbf{iterable} if there is a function
 \[iterator(x\in D[A])\in Iterator[A]\]

Now the imoprtance of iterators follows from two facts:
\begin{compactitem}
 \item Many data structures $D$ are iterable (see Sect.~\ref{sec:ad:iter:create}).
 \item Many important operations for $D$ can be realized using only the functionality of iterators (see Sect.~\ref{sec:ad:iter:use}).
\end{compactitem}
Thus, iterators provide a sweet-spot in the trade-off between simplicity and expressivity---they are very simple,  but we can do a lot with them.

\begin{remark}[Simplicity vs. Expressivity]
The trade-offs between simplicity and expressivity comes up again and again in computer science.
The best data structures combine both properties, but usually they are mutually exclusive.

All the important data structures presented in Part~\ref{sec:ad:ds} have become important because they do well in this way.
\end{remark}

An important function on iterators is $map$:
\begin{ctabular}{|l|l|}
\hline
function & returns \\
\hline
$foreach(x\in Iterator[A],f\in A\to B)\in Iterator[B]$ & an iterator for $[f(a_1),\ldots,f(a_n)]$ where $X=[a_1,\ldots,a_n]$ \\
\hline
\end{ctabular}

The trick behind $map$ is that $x$ is not traversed right away.
Instead, we create a new iterator that, when traversed, applies $f$.
That way we ensure that $f$ is applied only as often as necessary.

\subsection{Data Structure}

We can give a data structure for iterators as an abstract class:
\begin{acode}
\aclassA{Iterator[A]}{}{}{
 \afun[\Bool]{hasNext}{}{}
 \acomment{precondition for $getNext$ is $hasNext==\true$}\\
 \afun[A]{getNext}{}{}
}
\end{acode}

Then we can give an algorithm for $map$ as follows:
\begin{acode}
\aclass{Map[A,B]}{x:Iterator[A], f:A\to B}{Iterator[B]}{
  \afunI[\Bool]{hasNext}{}{x.hasNext}\\
  \afunI[B]{getNext}{}{f(x.getNext)}
}\\
\\
\afunI{map}{x:Iterator[A], f:A\to B}{\anew{Map[A,B]}{x,f}}
\end{acode}

\subsection{Working with Iterable Data Structures}\label{sec:ad:iter:use}

Let us assume an iterable data structure $D[A]$.
Our goal is to define functions on $x\in D[A]$ that use only $iterator(x)$.
There are indeed many of those.
Some important ones are:
\begin{ctabular}{|l@{}l@{}l|l|}
\hline
function &&& returns \\
\hline
\multicolumn{4}{|c|}{below, let $X=iterator(x)$}\\
$length$&$(x\in D[A])$&$\in \N$ & numbers of elements in $X$ \\
$contains$&$(x\in D[A],\; a\in A)$&$\in \Bool$ & $\true$ if $a$ occurs in $X$ \\
$index$&$(x\in D[A],\; a\in A)$&$\in \N^?$ & the position of the first occurrence of $a$ in $X$ (if any)\\
$find$&$(x\in D[A],\; p\in A\to\Bool)$&$\in A^?$ & the first element $a$ in $X$ (if any) such that $p(a)$ is $\true$ \\
$count$&$(x\in D[A],\; p\in A\to\Bool)$&$\in\N$ & the number of elements $a$ in $X$ for which $p(a)$ is $\true$ \\
$forall$&$(x\in D[A],\; p\in A\to\Bool)$&$\in \Bool$ & $\true$ if $p(a)$ is $\true$ for every element $a$ in $X$ \\
$exists$&$(x\in D[A],\; p\in A\to\Bool)$&$\in \Bool$ & $\true$ if $p(a)$ is $\true$ for some element $a$ in $X$ \\
$results$&$(x\in D[A],\;f\in A\to B)$&$\in List[B]$ & the list of results from applying $f$ to all $a$ in $X$ \\
$fold$&$(x\in D[A],\; b\in B, f\in A\times B\to B)$&$\in B$ & $f(a_1,f(a_2,\ldots,f(a_n,b))\ldots)$ with $X=[a_1,\ldots,a_n]$\\
\hline
\end{ctabular}
All of the above functions should not have a side-effect.
However, some of them take other functions as arguments.
It is usually a bad to do so, but it is technically possible that these functions have side-effects.
There is only one exception where we explicitly allow $f$ to have a side-effect:
\begin{ctabular}{|l|l|l|}
\hline
function & returns & effect \\
\hline
$foreach(x\in D[A],f\in A\to \Unit)\in \Unit$ & nothing & apply $f$ to all $a$ in $X$ \\
\hline
\end{ctabular}


\subsection{Making Data Structures Iterable}\label{sec:ad:iter:create}

Many important data structures are naturally iterable.

That includes in particular all data structures for lists:
\begin{acode}
\aclass{ListIterator[A]}{l: List[A]}{Iterator[A]}{
  index := 0\\
  \afunI[\Bool]{hasNext}{}{index < length(l)} \\
  \afun[A]{getNext}{}{
    a := get(l, index) \\
    index := index + 1 \\
    a
  }
}\\
\\
\afunI[{Iterator[A]}]{iterator}{l:List[A]}{\anew{ListIterator}{l}}
\end{acode}


\section{Streams}

$Stream[A]$ is not a data structure for the set $A^*$.
Instead, it is a data structure for the set $A^\N$.

The set $A^\N$ contains functions $f:\to A$, which we can think of as inifite lists $[f(0),f(1),\ldots]$.
Because they are so similar to lists, they are usually treated together with lists, even they do not realize the same set.

The set $A^\N$ is uncountable.
Therefore, not all possible streams are effective objects that can be represented in a physical machine.
However, for many practical purposes, it is fine to treat $Stream[A]$ as if it were the type of all possible streams.

$Stream[A]$ is usually implemented in the same way as $Iterator[A]$ with the understanding that $hasNext$ is always $\true$, i.e., the stream is never over.

Consequently, the functions on $Iterator[A]$ behave slightly differently when used for $Stream[A]$.
For exapmle:
\begin{compactitem}
 \item We cannot call $length$, $count$, $results$, $fold$, and $foreach$ on streams.
 \item We can call $contains$ on a stream. However, the function may run forever if the searched-for element is not in the stream.
 The same caveat applies to $index$, $find$, $forall$, and $exists$.
 \item We can call $map$. But it must be a special variant of $map$ that returns a new iterator without actually applying the map-function.
\end{compactitem}