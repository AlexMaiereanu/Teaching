\section{Specification}

After lists, trees are the next most important data structure in computer science.
They can be seen as a generalization of lists where the elements are not arranged in a row, but branching is allowed.

There are many equivalent definitions.
The easiest is by graphical example: A tree is something that looks like

\tikzstyle{node}=[circle,draw]
\begin{center}
\begin{tikzpicture}
\node[node] (0) at (0,0) {a};
\node[node] (00) at (-2,-1) {b};
\node[node] (01) at (0,-1) {c};
\node[node] (02) at (2,-1) {d};
\node[node] (000) at (-2,-2) {e};
\node[node] (010) at (-1,-2) {f};
\node[node] (011) at (1,-2) {g};
\draw[arrow] (0) -- (00);
\draw[arrow] (0) -- (01);
\draw[arrow] (0) -- (02);
\draw[arrow] (00) -- (000);
\draw[arrow] (01) -- (010);
\draw[arrow] (01) -- (011);
\end{tikzpicture}
\end{center}

\subsection{General Trees}

A more formal definition is this:

\begin{definition}[Tree]\label{def:ad:tree}
A \textbf{tree} is a connected directed graph in which
\begin{compactitem}
 \item there is exactly one node (called the \textbf{root}) with in-degree $0$,
 \item all other nodes have in-degree $1$.
\end{compactitem}
\end{definition}
Here we already used the more general concept of graphs, which we define formally in Sect.~\ref{sec:ad:graph}.

Talking about the shape and parts of a tree can be confusing.
Therefore, we introduce some vocabulary that helps us:

\begin{definition}[Parts of a Tree]\label{def:ad:treeaux}
For every edge from $p$ to $c$, we call $p$ the \textbf{parent} of $c$ and $n$ a \textbf{child} of $p$.
Thus, the root has no parent; every other node has exactly one parent.
A node may have any number of children.
A node with $0$ children is called a \textbf{leaf}.
A node that is neither the root nor a child is called an \textbf{inner node}.

For every path from $a$ to $d$, we call $a$ an \textbf{ancestor} of $d$ and $d$ a \textbf{descendant} of $a$.
Thus, all nodes are descendants of the root; every node is an ancestor/descendant of itself.
A \textbf{proper} ancestor/descendant of $n$ is an ancestor/descendant that is not $n$.

The number of proper ancestors of $n$ is called the \textbf{depth} of $n$.
Thus: The root has depth $0$.

A path from the root to a leaf is called a \textbf{branch}.
Thus, every leaf $l$ is part of exactly one branch, whose length is the depth of $l$.
The length of the longest branch(es) is called the \textbf{height} of the tree.
\end{definition}

Def.~\ref{def:ad:tree} only defines the abstract shape of trees.
But trees are only useful if we can store some data in each node.
The following definition does that:

\begin{definition}[Tree Over a Set]\label{def:ad:labeledtree}
The set $Tree[A]$ of \textbf{trees over the set} $A$ contains pairs $(T,f)$ where $T$ is a tree $T$ and $f$ is a function from nodes of $T$ to $A$.

If $f(n)=a$, we call $a$ the \textbf{label} of $n$.
\end{definition}

\begin{remark}[Leaf-Labeled Trees]
Occassionally, we are also interested in trees where only the leafs are labeled with.
And sometimes we need trees where inner nodes are labeled with elements of $A$ and leafs with elements of $B$.

We ignore those tree here.
But when working with someone else's tree data structures, it is important to check which nodes are labeled with what.
\end{remark}

\subsection{Binary Trees}

Binary trees are an important special case:

\begin{definition}[Binary Tree]\label{def:ad:bintree}
A \textbf{binary tree} is a tree in which all non-leaf nodes have exactly $2$ children.

Binary trees over a set are defined accordingly.

A binary tree is called \textbf{balanced} if the lengths of all branches differ by at most $1$.
Equivalently, the depth of all leafs differs by at most $1$.
It is called \textbf{perfectly balanced} if the lengths of all branches are equal.
\end{definition}

It is important to know the number of nodes in a binary tree:

\begin{theorem}\label{thm:ad:bintree}
A binary tree has at most $2^n$ nodes at depth $n$.
It has at most $2^{n+1}-1$ nodes in total.

If the tree has height $h$ and is balanced, it has
\begin{compactitem}
  \item exactly $2^n$ nodes at depth $n<h$,
  \item between $1$ and $2^h$ nodes at depth $h$,
  \item between $2^h$ and $2^{h+1}-1$ nodes in total. 
\end{compactitem}
If it is perfectly balanced, it has exactly $2^n$ nodes at any depth $n$ and exactly $2^{h+1}-1$ nodes in total.
\end{theorem}
\begin{proof}
Exercise.
\end{proof}

In particular, the number of nodes grows exponentially with the depth.

\section{Data Structures}
% as inductive type, pointer tree

\section{Important Algorithms}

\subsection{Depth-First Search}

\subsection{Breadth-Frist Search}

\subsection{Min-Max Algorithm}

\section{Heaps}\label{sec:ad:heaps}

\subsection{Specification}
% a binary tree is a heap if all branches are ordered

\subsection{Data Structures}

\subsection{Priority Queue}\label{sec:ad:heapqueue}

\subsection{Heapsort Algorithm}\label{sec:ad:heapsort}

\section{Search Trees}

Binary search trees and red-black trees are discussed in Sect.~\ref{sec:ad:sets}.